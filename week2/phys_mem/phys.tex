\begin{frame}
\frametitle{Физическая и логическая память}
\begin{itemize}
    \item<1-> Логическая память - как видит память программа.
    \item<2-> Физическая память - как процессор видит память
    \begin{itemize}
        \item<3-> логические адреса отображаются на физические;
        \item<4-> все что соблюдает интерфейс выглядит как \\
        физическая память.
    \end{itemize}
\end{itemize}
\end{frame}

\begin{frame}
\frametitle{Физическая память}
\begin{itemize}
    \item<1-> Не все физические адреса соответствуют памяти
    \begin{itemize}
        \item<1-> видеобуфер и другие устройства;
        \item<2-> "дыры".
    \end{itemize}
    \item<3-> Физическое адресное пространство может быть \\
    устроено довольно сложно
    \begin{itemize}
        \item без карты памяти не разобраться.
    \end{itemize}
\end{itemize}
\end{frame}

\begin{frame}
\frametitle{Карта физической памяти}
\begin{itemize}
    \item<1->Откда брать карту физической памяти?
    \begin{itemize}
        \item<2->из документации аппартной платформы;
        \item<3->спросить BIOS/UEFI/etc:
        \begin{itemize}
            \item BIOS: прерывание 0x15, команда 0xe820;
            \item UEFI: функция GetMemoryMap;
        \end{itemize}
        \item<4->спросить загрузчик
        \begin{itemize}
            \item GRUB поддерживает спецификацию multiboot.
        \end{itemize}
    \end{itemize}
\end{itemize}
\end{frame}
