\begin{frame}
\frametitle{Планировщик Windows}
\begin{itemize}
    \item<1->Потокам в Windows назначен приоритет
    \begin{itemize}
        \item<1->приоритет состоит из класса и приоритета внутри класса.
    \end{itemize}
    \item<2->На исполнение выбирается поток с наивысшим приоритетом
    \begin{itemize}
        \item<3->для потоков с равными приоритетами используется RR.
    \end{itemize}
\end{itemize}
\end{frame}

\begin{frame}
\frametitle{Priority Boost}
\begin{itemize}
    \item<1->Чтобы избежать неограниченного голодания потоков Windows иногда
         повышает им приоритет:
    \begin{itemize}
        \item<2->если поток отвечает за видимую часть UI;
        \item<3->при получении ввода или завершении операции IO;
        \item<4->для "случайно" выбранных потоков.
    \end{itemize}
\end{itemize}
\end{frame}


\begin{frame}
\frametitle{Планировщик Linux (один из)}
\begin{itemize}
    \item<1->Completely Fair Scheduler (CFS) - честный планировщик:
    \begin{itemize}
        \item<2->для каждого потока поддерживается "виртуальное время";
        \item<3->"виртуальное время" увеличивается, когда поток работает;
        \item<4->CPU отдаем потоку с наименьшим "виртуальным временем".
    \end{itemize}
\end{itemize}
\end{frame}
