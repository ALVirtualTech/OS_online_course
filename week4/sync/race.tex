\begin{frame}[fragile]
\frametitle{Пример}
    \begin{columns}
        \begin{column}{.4\textwidth}
            \begin{lstlisting}
    .data
counter:
    .int 0

    .text
add_one:
    movq counter, %rax
    inc %rax
    movq rax, counter
    retq
            \end{lstlisting}
        \end{column}
        \begin{column}{.4\textwidth}
            \begin{lstlisting}
int counter;

int add_one(void)
{
    return ++counter;
}	
            \end{lstlisting}
        \end{column}
        \begin{column}{.2\textwidth}
        \end{column}
    \end{columns}
\end{frame}

\begin{frame}[fragile]
\frametitle{Вариант 1}
\begin{columns}
    \begin{column}{.35\textwidth}
        \begin{lstlisting}
movq counter, %rax
inc %rax
movq rax, counter
 
 
 
        \end{lstlisting}
    \end{column}
    \begin{column}{.35\textwidth}
        \begin{lstlisting}
 
 
 
movq counter, %rax
inc %rax
movq rax, counter
        \end{lstlisting}
    \end{column}
    \begin{column}{.3\textwidth}
    \end{column}
\end{columns}
\end{frame}

\begin{frame}[fragile]
\frametitle{Вариант 2}
\begin{columns}
    \begin{column}{.35\textwidth}
        \begin{lstlisting}
movq counter, %rax

inc %rax
movq rax, counter
 
 
        \end{lstlisting}
    \end{column}
    \begin{column}{.35\textwidth}
        \begin{lstlisting}
 
movq counter, %rax
 
 
inc %rax
movq rax, counter
        \end{lstlisting}
    \end{column}
    \begin{column}{.3\textwidth}
    \end{column}
\end{columns}
\end{frame}

\begin{frame}
\frametitle{Состояние гонки}
\begin{itemize}
    \item<1->Состояние гонки - результат зависит от порядка выполнения
         инструкций
    \begin{itemize}
        \item<2->порядок зависит от слишком многих факторов;
        \item<3->решения планировщика, влияние других потоков, прерывания...
        \item<4->могут быть трудно воспроизводимы - не поддаются тестированию.
    \end{itemize}
\end{itemize}
\end{frame}

\begin{frame}
\frametitle{Критическая секция}
\begin{itemize}
    \item<1->Критическая секция
    \begin{itemize}
        \item<2->участок кода обращающийся к разделяемым несколькими потоками
             данным;
        \item<3->если не более чем один поток может одновременно находится в
             критической секции, то не будет состояния гонки.
    \end{itemize}
\end{itemize}
\end{frame}

\begin{frame}
\frametitle{Блокировка}
\begin{itemize}
    \item<1->Блокировка (lock) - некоторый объект и пара методов для работы с
         ним
    \begin{itemize}
        \item<2->lock - метод захвата блокировки;
        \item<3->unlock - метод освобождения блокировки.
    \end{itemize}
\end{itemize}
\end{frame}

\begin{frame}
\frametitle{Свойство взаимного исключения}
\begin{itemize}
    \item<1->Взимное исключение (mutual exclusion)
    \begin{itemize}
        \item<2->потоки всегда вызывают lock и unlock парами (сначала lock,
             а потом unlock);
        \item<3->не более одного потока может одновременно находиться между
             lock-ом и unlock-ом.
    \end{itemize}
\end{itemize}
\end{frame}

\begin{frame}[fragile]
\frametitle{Свойство взаимного исключения}
\begin{lstlisting}
    struct lock l;
    int counter;

    int add_one(void)
    {
        int res;

        lock(&l);
        res = ++counter;
        unlock(&l);
    }
\end{lstlisting}
\end{frame}

\begin{frame}[fragile]
\frametitle{Свойство взаимного исключения}
\begin{columns}
    \begin{column}{.4\textwidth}
        \begin{lstlisting}
struct lock lock0;
int counter0;

int add_one0(void)
{
    int res;

    lock(&lock0);
    res = ++counter0;
    unlock(&lock0);
    return res;
}
        \end{lstlisting}
    \end{column}
    \begin{column}{.4\textwidth}
        \begin{lstlisting}
struct lock lock1;
int counter1;

int add_one1(void)
{
    int res;

    lock(&lock1);
    res = ++counter1;
    unlock(&lock1);
    return res;
}
        \end{lstlisting}
    \end{column}
    \begin{column}{.2\textwidth}
    \end{column}
\end{columns}
\end{frame}

\begin{frame}[fragile]
\frametitle{Свойство живости}
\begin{lstlisting}
    struct lock {
    };

    void lock(struct lock *unused)
    {
        (void) unused;
        while (1);
    }

    void unlock(struct lock *unused)
    {
        (void) unused;
    }
\end{lstlisting}
\end{frame}

\begin{frame}
\frametitle{Свойство живости}
\begin{itemize}
    \item<1->Свойство живости (deadlock freedom)
    \begin{itemize}
        \item<2->если один из потоков вызвал lock, то какой-то из
             потоков вызвавших lock захватит блокировку;
        \item<3->поток не ждет в lock, если он единственный пытается захватить
             блокировку;
        \item<4->если поток ждет, значит другому потоку повезло захватить
             блокировку.
    \end{itemize}
\end{itemize}
\end{frame}

\begin{frame}
\frametitle{На что нельзя полагаться?}
\begin{itemize}
    \item<1->Скорость работы потоков:
    \begin{itemize}
        \item<2->мы не знаем сколько времени потребуется потоку, чтобы
             выполнить какой-то код;
        \item<3->мы не можем полагать, что какой-то поток быстрее.
    \end{itemize}
\end{itemize}
\end{frame}

\begin{frame}
\frametitle{На что можно полагаться?}
\begin{itemize}
    \item<1->Потоки работают корректно:
    \begin{itemize}
        \item<2->поток не находится между lock и unlock бесконечно;
        \item<3->поток не "падает" находясь между lock и unlock;
        \item<4->и так далее...
    \end{itemize}
\end{itemize}
\end{frame}
